\documentclass{ltjsarticle}


%ソースコード
\usepackage{listings}
\lstset{
  numbers=left,
  basicstyle=\ttfamily,
}
%装飾
\usepackage{color}
% 数式
\usepackage{amsmath,amssymb}
\usepackage{bm}
\usepackage{physics}
\usepackage{comment}
\usepackage{autobreak}
\usepackage{mathtools}
\usepackage{mathcommand}
\mathtoolsset{showonlyrefs=true}
% 数式処理
\usepackage{luacas}
% 画像
\usepackage{graphicx}
\usepackage{here}
\usepackage{tikz}
% 引用
\usepackage{hyperref}

\title{院試メモ}
\author{Toroid153846}
\date{\today}

\begin{document}
\maketitle
  \section{試験前準備}
  消しゴム忘れない\\
  腕時計忘れない
  \section{一般}
  次元考える\\
  証明問題で正しいことがほぼ確定していたら問題に丸をつける\\
  抽象的な問題で複雑な計算が出てきそうでもとりあえず手を動かして具体化する\\
  わかるところがなくなったと思っても見直しのついでに再計算してみるとうまくいくことがある\\
  誘導がある時、その式をそのまま使うという場合以外にも具体的な値を代入することで用いるという場合もある\\
  問題の途中から条件が変わることがあれば忘れないようにきちんと強調しておく\\
  その問題以降も使える条件があるなら強調しておく\\
  連問で解けなそうだと思っても次の問題はあまり関係がない場合もある\\
  なんか見覚えがあることは危険信号なので一旦飛ばす\\
  きちんとどの問題で記述が必要かどうか確認する\\
  怪しいと思っても時間あればとりあえず計算しておく\\
  計算が多すぎておかしいと思ったら問題の条件をきちんと確認する\\

  \section{物理数学}
  $y'=f\left( \frac{y}{x} \right) $のような微分方程式も$u=\frac{y}{x}$と置いて
  \begin{align}
    u'=\frac{xy'-y}{x^2}=\frac{y'-u}{x}=\frac{f(u)-u}{x}
  \end{align}
  とすれば変数分離により解ける。\\
  $y''=f(y)$のような微分方程式も$t=y'$と置いて
  \begin{align}
    y''=\dv{t}{x}=\dv{t}{y}\dv{y}{x}=\dv{t}{y}t
  \end{align}
  とすれば変数分離により解ける。\\
  微分方程式の二次方程式で重解$\alpha$が出た時は$\exp\left( \alpha x \right) $、$x \exp\left( \alpha x \right) $が二つの線型独立な解になる。\\
  三角関数双曲線関数の逆関数の微分(特に$\dv{x}\arctan{x}=\frac{1}{1+x^2}$で$\dv{x}\mathrm{arctanh}{x}=\frac{1}{1-x^2}$)\\
  双曲線関数のグラフの形覚える($\lim_{x\to\pm\infty}\tanh{x}=\pm 1$)\\
  次数が小さい場合には行列も具体的な形を考えてみる(分からない行列要素を置いてみたり)\\
  留数定理は一回Resをlimで書けば微分係数の逆数になってるとわかるのでそれで計算\\
  エルミート行列の累乗のトレース求めるだけなら、固有値の累乗の総和になる\\
  無限区間の場合にはフーリエ変換、有限区間の場合にはフーリエ級数を用いる\\
  固有ベクトルごとに考える\\
  存在することを示せ、と言う時に存在するものを作ってもいい\\
  留数定理の留数はまず何位の極かを求めてn位だとする
  \begin{align}
    \mathrm{Res}_{z=z_0}f(z)=\frac{1}{(n-1)!}\lim_{z\to z_0}\dv[n-1]{z}\left((z-z_0)^nf(z)\right)
  \end{align}
  1位の極ならば
  \begin{align}
    \mathrm{Res}_{z=z_0}f(z)&=\lim_{z\to z_0}(z-z_0)f(z)\\
    &=\lim_{z\to z_0}\frac{1}{\frac{\frac{1}{f(z)}-\frac{1}{f(z_0)}}{z-z_0}}\\
    &=\frac{1}{\eval{\dv{z}\frac{1}{f(z)}}_{z=z_0}}\\
  \end{align}
  と計算できる。\\
  複素積分の変数変換は簡単だと思ってもちゃんとやりましょう\\
  一応覚えておく
  \begin{align}
    \int_{-\infty}^{\infty}e^{-x^2}dx=\sqrt{\pi}
  \end{align}


  \section{物理}
  高校物理のような計算をする際には、きちんと$\pm$を気にすること。\\
  \subsection{古典力学}
  ラグランジアンのポテンシャルの$\pm$気をつける\\
  ラグランジアンは力学変数部分の同類項をまとめた方が運動方程式を立てる時に楽\\
  角運動量ベクトル$\bm{L}$、角速度ベクトル$\bm{\Omega}$として、トルク$\dv{\bm{L}}{t}=\bm{\Omega}\times\bm{L}$

  \subsection{電磁気学}
  $\nabla r=\frac{\bm{r}}{r}$、$\nabla f(r)=\frac{\bm{r}}{r}\dv{f}{r}$ぐらいは覚えとく\\
  原点に静止している点電荷qのポテンシャルは$\phi(\bm{r})=\frac{1}{4\pi\epsilon_0}\frac{q}{r}$、電場は$\frac{1}{4\pi\epsilon_0}\frac{q\bm{r}}{r^3}$\\
  スカラーポテンシャル$\phi$とベクトルポテンシャル$\bm{A}$を用いて、$\bm{E}=-\nabla\phi-\pdv{\bm{A}}{t},\bm{B}=\nabla\times \bm{A}$となる\\
  波動方程式の導出は面倒がらずにやっておく\\
  電磁波を扱うときは複素数$\exp(ikx)$の方がやりやすい(特に電磁場の浸透)\\
  \subsection{熱力学}
  $d'Q+\mu dN=dU+d'W$で準静的なら$d'Q=TdS$、$d'W=PdV$が成り立つ\\
  ギブス・デュエムの関係式は$VdP=SdT+Nd\mu$である。\\
  ヘルムホルツの自由エネルギーは$F=U-TS$、エンタルピーは$H=U+PV$、ギブスの自由エネルギーは$G=U+PV-TS$である。\\
  ギブスの自由エネルギーはP-T図で考えて相共存の時に等しくなる\\
  \subsection{統計力学}
  スピンの計算の場合にはきちんと$\sinh,\cosh,\tanh$を用いること。\\
  カノニカル分布$Z(T,V,N)=\sum_i \exp(-\beta E_i)$とグランドカノニカル分布$\Xi(T,V,\mu)=\sum_{i,N}\exp(-\beta(E_i-N\mu))=\sum_N Z(T,V,N)\exp(\beta N\mu)$の表式を覚える\\
  ボーズ分布$(\exp(\beta(\epsilon-\mu))-1)^{-1}$とフェルミ分布$(\exp(\beta(\epsilon-\mu))+1)^{-1}$の表式を覚える\\
  分配関数の計算では双曲線関数を用いた方が計算が楽になる\\
  分配関数の$\frac{1}{N!}$を忘れない(量子系だといらない。そもそもの量子数に入っている)\\
  磁化の表式は$-\pdv{F}{H}=$($\mu S$の合計)\\
  高温においては$\beta$、低温においては$\exp(-\beta)$でテイラー展開\\
  量子スピンなら単純に計算してはいけない(古典スピンと区別してシングレットトリプレットを考える)\\
  エルミート行列の累乗のトレース求めるだけなら、固有値の累乗の総和になる\\
  ボース理想気体においては化学ポテンシャルは基底エネルギーより小さい(ボースアインシュタイン凝縮においては基底エネルギーと等しくなる)\\

  \subsection{量子力学}
  束縛条件は無限遠点におけるポテンシャルの値よりもエネルギー固有値が低いこと。\\
  エルミート共役を取る時に複素共役取るの忘れない\\
  運動量表示$\phi(p)$として
  \begin{align}
    \phi(p)
    &=\bra{p}\ket{\phi}\\
    &=\bra{p}\left( \int^\infty_{-\infty}\ketbra{x} \right) \ket{\phi}\\
    &=\int^\infty_{-\infty}\bra{p}\ket{x}\bra{x}\ket{\phi}dx\\
    &=\frac{1}{\sqrt{2\pi}}\int^\infty_{-\infty}\exp\left( -\frac{ipx}{\hbar} \right)\bra{x}\ket{\phi}dx\\
  \end{align}
  と表せる。\\
  ハミルトニアンかける時にエネルギー固有状態があるならその基底の表示でかけましょうね\\
  一次元一粒子の時間依存無しシュレディンガー方程式解においては$n$番目の励起状態において$\delta x\cdot \delta p\simeq n\hbar$となる\\
  方位量子数$s$、磁気量子数$m$として
  \begin{align}
    S^z=\hbar m\ket{s,m}\\
    S^\pm\ket{s,m}=(S^x\pm iS^y)\ket{s,m}=\hbar\sqrt{s(s+1)-m(m\pm 1)}\ket{s,m\pm 1}\\
    \bm{S}^2\ket{s,m}=\left((S^z)^2+\frac{1}{2}(S^+S^-+S^-S^+)\right)\ket{s,m}=\hbar^2s(s+1)\ket{s,m}\\
  \end{align}
  ぐらいは覚えておく\\
  $H=H_0+H'$として、摂動論の公式
  \begin{align}
    E_n^{(1)}&=\bra{\psi_n^{(0)}}H' \ket{\psi_n^{(0)}}\\
    E_n^{(2)}&=\sum_{m(\neq n)}\frac{\abs{\bra{\psi_m^{(0)}}H' \ket{\psi_n^{(0)}}}^2}{E_n^{(0)}-E_m^{(0)}}
  \end{align}
  などを覚えておく。\\
  ボーア=ゾンマーフェルト量子条件(半古典近似)
  \begin{align}
    \oint p(x)dx=nh
  \end{align}




\end{document}
