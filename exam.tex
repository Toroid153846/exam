\documentclass{ltjsarticle}


%ソースコード
\usepackage{listings}
\lstset{
  numbers=left,
  basicstyle=\ttfamily,
}
%装飾
\usepackage{color}
% 数式
\usepackage{amsmath,amssymb}
\usepackage{bm}
\usepackage{physics}
\usepackage{comment}
\usepackage{autobreak}
\usepackage{mathtools}
\usepackage{mathcommand}
\mathtoolsset{showonlyrefs=true}
% 数式処理
\usepackage{luacas}
% 画像
\usepackage{graphicx}
\usepackage{here}
\usepackage{tikz}
% 引用
\usepackage{hyperref}

\title{院試メモ}
\author{Toroid153846}
\date{\today}

\begin{document}
\maketitle


  \section{数学}
  \subsection{線形代数}
  \subsection{微積分}
  \subsection{微分方程式}
  \subsection{複素積分}
  \subsection{フーリエ変換}
  \subsection{ラプラス変換}
  \section{物理}
  \subsection{古典力学}
  \subsection{電磁気学}
  \subsection{熱力学}
  \subsection{量子力学}
  束縛条件は無限遠点におけるポテンシャルの値よりもエネルギー固有値が低いこと。\\
  \subsection{統計力学}
  磁化の計算は$\mu S_z$の平均値を計算すること。\\
  スピンの計算の場合にはきちんと$\sinh,\cosh,\tanh$を用いること。\\



\end{document}
